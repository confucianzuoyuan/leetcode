\documentclass[oneside]{ctexbook}
\usepackage{hyperref}
\hypersetup{
    colorlinks,
    citecolor=black,
    filecolor=black,
    linkcolor=black,
    urlcolor=black
}
\usepackage{tcolorbox}
\tcbuselibrary{most}

\newtcolorbox{mytext}[2][]{
    title=#2,
    colback=red!5!white,
    colframe=red!75!black,
    fonttitle=\bfseries,
    colbacktitle=red!85!black,
    enhanced,
    halign=center,
    attach boxed title to top center={yshift=-2mm},
    #1
}

\title{神曲}
\author{但丁}

\begin{document}

\maketitle

\tableofcontents

\chapter{}

\begin{mytext}{1-3}
Midway in the journey of our life

I came to myself in a dark wood,

for the straight way was lost.
\end{mytext}

\begin{mytext}{4-6}
Ah, how hard it is to tell

the nature of that wood, savage, dense and harsh-

the very thought of it renews my fear!
\end{mytext}

\begin{mytext}{7-9}
It is so bitter death is hardly more so.

But to set forth the good I found

I will recount the other things I saw.
\end{mytext}

\begin{mytext}{10-12}
How I came there I cannot really tell,

I was so full of sleep

when I forsook the one true way.

\tcblower

forsake: abandon
\end{mytext}

\begin{mytext}{13-15}
But when I reached the foot of a hill,

there where the valley ended

that had pierced my heart with fear,
\end{mytext}

\begin{mytext}{16-18}
looking up, I saw its shoulders

arrayed in the first light of the planet

that leads men straight, no matter what their road.

\tcblower

array: 排列
\end{mytext}

\begin{mytext}{19-21}
Then the fear that had endured

in the lake of my heart, all the night

I spent in such distress, was calmed.

\tcblower

distress: extreme anxiety, sorrow, or pain.
\end{mytext}

\begin{mytext}{22-24}
And as one who, with laboring breath,

has escaped from the deep to the shore

turns and looks back at the perilous waters,

\tcblower

perilous: full of danger or risk
\end{mytext}

\begin{mytext}{25-27}
so my mind, still in flight,

turned back to look once more upon the pass

no mortal being ever left alive.

\tcblower

mortal: 
\end{mytext}

\begin{mytext}{28-30}
After I rested my wearied flesh a while,

I took my way again along the desert slope,

my firm foot always lower than the other.

\tcblower

fless:

desert:

slope:
\end{mytext}

\begin{mytext}{31-33}
But now, near the beginning of the steep,

a leopard light and swift

and covered with a spotted pelt

\tcblower

steep:

leopard:

spotted:

pelt:
\end{mytext}

\begin{mytext}{34-36}
refused to back away from me

but so impeded, barred the way,

that many times I turned to go back down.

\tcblower

impeded:

barred:
\end{mytext}

\begin{mytext}{37-39}
It was the hour of morning,

when the sun mounts with those stars

that shone with it when God's own love

\tcblower

mount:
\end{mytext}

\begin{mytext}{40-42}
first set in motion those fair things,

so that, despite that beast with gaudy fur,

I still could hope for good, encouraged

\tcblower

gaudy:
\end{mytext}

\begin{mytext}{43-45}
by the hour of the day and the sweet season,

only to be struck by fear

when I beheld a lion in my way.

\tcblower

behold:
\end{mytext}

\begin{mytext}{46-48}
He seemed about to pounce—

his head held high and furious with hunger—

so that the air appeared to tremble at him.

\tcblower

pounce:

furious:

tremble:
\end{mytext}

\begin{mytext}{49-51}
And then a she-wolf who, all hide and bones,

seemed charged with all the appetites

that have made many live in wretchedness

\tcblower

appetite:

wretchedness:
\end{mytext}

\begin{mytext}{52-54}
so weighed my spirits down with terror,

which welled up at the sight of her,

that I lost hope of making the ascent.

\tcblower

well up:

ascent:
\end{mytext}

\begin{mytext}{55-57}
And like one who rejoices in his gains

but when the time comes and he loses,

turns all his thought to sadness and lament,

\tcblower

rejoice:

lament:
\end{mytext}

\begin{mytext}{58-60}
such did the restless beast make me—

coming against me, step by step,

it drove me down to where the sun is silent.
\end{mytext}

\begin{mytext}{61-63}
While I was fleeing to a lower place,

before my eyes a figure showed,

faint, in the wide silence.
\end{mytext}

\begin{mytext}{64-66}
When I saw him in that vast desert,

'Have mercy on me, whatever you are,'

I cried, 'whether shade or living man!'
\end{mytext}

\begin{mytext}{67-69}
He answered: 'Not a man, though once I was.

My parents were from Lombardy—

Mantua was their homeland.
\end{mytext}

\begin{mytext}{70-72}
'I was born \textit{sub Julio}, though late in his time,

and lived at Rome, under good Augustus

in an age of false and lying gods.
\end{mytext}

\begin{mytext}{73-75}
'I was a poet and I sang

the just son of Anchises come from Troy

after proud Ilium was put to flame.
\end{mytext}

\begin{mytext}{76-78}
'But you, why are you turning back to misery?

Why do you not climb the peak that gives delight,

origin and cause of every joy?'
\end{mytext}

\begin{mytext}{79-81}
'Are you then Virgil, the fountainhead

that pours so full a stream of speech?'

I answered him, my head bent low in shame.
\end{mytext}

\begin{mytext}{82-84}
'O glory and light of all other poets,

let my long study and great love avail

that made me delve so deep into your volume.
\end{mytext}

\begin{mytext}{85-87}
'You are my teacher and my author.

You are the one from whom alone I took

the noble style that has brought me honor.
\end{mytext}

\begin{mytext}{88-90}
'See the beast that forced me to turn back.

Save me from her, famous sage—

she makes my veins and pulses tremble.'
\end{mytext}

\begin{mytext}{91-93}
'It is another path that you must follow,'

he answered, when he saw me weeping,

'if you would flee this wild and savage place.
\end{mytext}

\begin{mytext}{94-96}
'For the beast that moves you to cry out

lets no man pass her way,

but so besets him that she slays him.
\end{mytext}

\begin{mytext}{97-99}
'Her nature is so vicious and malign

her greedy appetite is never sated—

after she feeds she is hungrier than ever.
\end{mytext}

\begin{mytext}{100-102}
'Many are the creatures that she mates with,

and there will yet be more, until the hound

shall come who'll make her die in pain.
\end{mytext}

\begin{mytext}{103-105}
'He shall not feed on lands or lucre

but on wisdom, love, and power.

Between felt and felt shall be his birth.
\end{mytext}

\begin{mytext}{106-108}
'He shall be the salvation of low-lying Italy,

for which maiden Camilla, Euryalus,

Turnus, and Nisus died of their wounds.
\end{mytext}

\begin{mytext}{109-111}
'He shall hunt the beast through every town

till he has sent her back to Hell

whence primal envy set her loose.
\end{mytext}

\begin{mytext}{112-114}
'Therefore, for your sake, I think it wise

you follow me: I will be your guide,

leading you, from here, through an eternal place
\end{mytext}

\begin{mytext}{115-117}
'where you shall hear despairing cries

and see those ancient souls in pain

as they bewail their second death.
\end{mytext}

\begin{mytext}{118-120}
'Then you shall see the ones who are content

to burn because they hope to come,

whenever it may be, among the blessed.
\end{mytext}

\begin{mytext}{121-123}
'Should you desire to ascend to these,

you'll find a soul more fit to lead than I:

I'll leave you in her care when I depart.
\end{mytext}

\begin{mytext}{124-126}
'For the Emperor who has His seat on high

wills not, because I was a rebel to His law,

that I should make my way into His city.
\end{mytext}

\begin{mytext}{127-129}
'In every part He reigns and there He rules.

There is His city and His lofty seat.

Happy the one whom He elects to be there!'
\end{mytext}

\begin{mytext}{130-132}
And I answered: 'Poet, I entreat you

by the God you did not know,

so that I may escape this harm and worse,
\end{mytext}

\begin{mytext}{133-135}
'lead me to the realms you've just described

that I may see Saint Peter's gate

and those you tell me are so sorrowful.'
\end{mytext}

\begin{mytext}{136}
Then he set out and I came on behind him.
\end{mytext}

\chapter{}

\end{document}